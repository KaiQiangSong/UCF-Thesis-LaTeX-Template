%%%%%%%%%%%%%%%%%%%%%%%%%%%%%%%%%%%%%%%%%%%%%%%%%%%%%%%%
%
% UCF Electronic Dissertation and Thesis (ETD) Template
% (C) Copyright 2007-2013 Daniel Gallagher,
%
% https://bitbucket.org/dgallagher/ucf-thesis-latex-template
%
% PRELIMINARY PAGES, ACKNOWLEDGMENTS, AND ABSTRACT
%
%%%%%%%%%%%%%%%%%%%%%%%%%%%%%%%%%%%%%%%%%%%%%%%%%%%%%%%%%%%%%%%%%%%%%%
%
\title       {UCF~ETD~Title~Here}
\author      {Your~Name~Here}
%
% ~~~~~~~~~~~~~~~~~~~~~~~~~~~~~~
\prevdegreei    {B.S. University of Central Florida, 2004}
\noprevdegreeii   %{Professional Degree of Electronics Engineer, 1995}
\noprevdegreeiii  %{M.S. University of Central Florida, 2000}
% ~~~~~~~~~~~~~~~~~~~~~~~~~~~~~~
\department {Electrical and Computer Engineering}
\school     {Electrical Engineering and Computer Science}
\college    {Engineering and Computer Science}
\term       {Summer}
% Note:  degreeyear should be optional, but as of  5-Feb-96
% it seems required or you get a year of ``2''.   -johnh
\degreeyear {2007}
\professor  {John A.\ Professor}
% ~~~~~~~~~~~~~~~~~~~~~~~~~~~~~~
\nosigpage
\defensedate {July 13, 2007} % For Signature Page
%\nocopyright % Uncomment for no copyright page
% ~~~~~~~~~~~~~~~~~~~~~~~~~~~~~~
% For Masters Defense, Chair plus 2 committee members required
\chair      {John A.\ Professor}
\member     {Committee W.\ Member}
\member     {Committee G.\ Member}
\member     {Committee A.\ Member}
% ~~~~~~~~~~~~~~~~~~~~~~~~~~~~~~
\dedication     {\textsl{This thesis is dedicated to Daniel Gallagher for giving me his \LaTeX\ template which he so graciously published as a Git repository at: \url{https://bitbucket.org/dgallagher/ucf-thesis-latex-template}.}}
% ~~~~~~~~~~~~~~~~~~~~~~~~~~~~~~
\acknowledgments {
Include any desired acknowledgements here.
}
% ~~~~~~~~~~~~~~~~~~~~~~~~~~~~~~
 \sloppy
% \fussy
% \relax
% ~~~~~~~~~~~~~~~~~~~~~~~~~~~~~~
\abstract{ % Thesis abstract goes here or included in separate file for simplicity.
\lipsum[1] % Placeholder text.
%\input{abstract}
}
% ~~~~~~~~~~~~~~~~~~~~~~~~~~~~~~
% \acronyms{ % Optional Section requiring additional packages or macros (such as notations)
%\input{acronyms}
%}
